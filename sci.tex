% sci.tex - SCI STYLE FILE 使用例

% pdfLatex
%
% 日本語原稿の場合
%\documentclass{bxjsarticle}
\documentclass{jarticle}
\usepackage[whole]{bxcjkjatype}
\usepackage{sci}
%\usepackage{amsmath,amssymb}
\renewcommand{\baselinestretch}{1.17}
%
% 英語原稿の場合
%\documentclass{article}
%\usepackage[whole]{bxcjkjatype}
%\usepackage{sci}
%\usepackage{amsmath,amssymb}
%\english

%=========================================
% LaTeX2e
% 
% 日本語原稿の場合
% \documentclass{jarticle}
% \usepackage{sci}
% 
%-----------------------------------------
% 英語原稿の場合
% \documentclass{article}
% \usepackage{latexsym}
% \english

% LaTeX209
% 
% 日本語原稿の場合
% \documentstyle[sci]{jarticle}
%
% 英語原稿の場合
% \documentstyle[sci]{article}
% \english

% 2018-12-21 modification by Sei Ikeda %SCI19
% 2006-11-16 modification by TF %SCI07

\jtitle{システム制御情報研究発表講演会用スタイルファイル}
\etitle{Style File for ISCIE Conferences}
\jauthor{著者の所属 ~~~ ○ 著者 \ 氏名,共著 ~ 者}
\eauthor{A. U. Thor \ and \ C. O. Author\\ 
  Affiliation of the author(s)}
\englishabstract{This is a sample document which uses 
  the ISCIE style file.
  Please use the keyword `englishabstract', not to be
  confused with the predefined `abstract' environment.
  The quick brown fox jumps over a lazy dog's back
  And so on \ldots}

\begin{document}

\maketitle

\section{はじめに}

システム制御情報学会研究発表講演会(SCI)用のスタイルファイル 
\verb+sci.sty+ は ASCII版の p\LaTeX\ を想定して作られており,
それ以外での動作は未確認です.このスタイルファイルは 
p\LaTeX2$\varepsilon$ および p\LaTeX 2.09 に対応しております.
%
% 2019年度追加
加えて,Overleaf v2上の pdf\LaTeX\ および p\LaTeX\ で動作するように
本サンプル\verb+sci.tex+ では,プリアンブル部分および
本文中の一部の記述が変更されています.

ファイル \verb+sci.sty+ には漢字コードが含まれています.動作
環境により漢字コードの扱いが異なりますので,場合によってはコ
ード変換の必要があることにご注意ください.

サンプルファイル \verb+sci.tex+ は \verb+sci.sty+ の使用例に
なっていますので,題目,著者名などの書き方の参考例としてお使
いいただけます.

基本的に,\LaTeX で原稿を作成する場合,1行あたりの文字数や,
1ページあたりの文字数を正確に見積もることは困難ですので,執
筆要項にある文字数,行数はあくまでも一つの目安とお考えくださ
い.ただし,講演論文集としてCD-ROMに収録する際に支障が生じな
いよう,上下左右の余白は執筆要項で指定された値以上を確保する
ようにご注意ください.

以下は,\verb+sci.sty+ の補足説明ですが,これはユーザマニュ
アルのようなものを意図したものではなく,むしろテクニカルマニ
ュアルに近いものとお考えください.

なお,2001年度より,従来のA4サイズからB5サイズへの縮小を取り
止め,A4サイズ原寸のままで印刷するすることになりました.さら
に,2007年度からは論文集を印刷せずにCD-ROMに収録することとな
りました.この変更により \verb+sci.sty+ は以前の
\verb+iscie.sty+ から大幅に変更されています.2000年度以前の 
\verb+iscie.sty+ は使用できませんのでご注意ください.
%
% 2019年度追加
2019年度より予稿集は主にWebダウンロードでの配布に切り替わりました.

以下は,theorem 環境の使用例です.

\begin{theorem}
ここに定理の内容を記述して下さい.系や補題の場合も同様です.
\end{theorem}
\begin{proof}
ここには定理の証明を記述して下さい.証明の最後には印がつきま
す.
\end{proof}

定理などの文章は,もともとイタリック書体を使うようになってい
ますが,和文との整合性を考えて,ローマン書体を使うように変更
しています.

\section{数式関係}
\verb+\eqnarray+ を使うと,等号(とは限りませんが)の両側の
スペースが広すぎるように感じられたので,この間隔を変更してい
ます.やはり,必要に応じて \verb+\eqnarray+ を再定義している
部分を変更あるいは削除してください.

以下は,eqnarray 環境の使用例です.
\begin{eqnarray}
 \dot{x}(t) & = & A x(t) + B u(t) \\
 y(t) & = & C x(t) + D u(t)
\end{eqnarray}

\section{英語原稿}

英語原稿の場合は \verb+jarticle+ ではなく \verb+article+ を
使い,プリアンブルで \verb+\english+ を忘れずに指定してくだ
さい.また,\verb+\usepackage{latexsym}+ も指定してください
(\LaTeX 209 の場合を除く).

クラス/スタイルオプションファイル \verb+sci.sty+ に漢字コー
ドが含まれていますので,英語原稿の場合でも日本語のp\LaTeX\ 
環境が必要です.ただし,原稿を記述した \verb+.tex+ ファイル
や整形後の \verb+.dvi+ ファイルに漢字コードが含まれないよう
にすることは可能です.

英語原稿では必ずしも口頭発表者の氏名の前に○印をつけることが
要求されていないようですが,印をつける場合
には \verb+{\LARGE $\circ$}+ のように入力してもよいと思われ
ます.

\section{その他}

スタイルファイル中ではPlain \TeX\ の \verb+\def+ および 
\LaTeX\ の \verb+\newcommand+, \verb+\renewcommand+ が混在し
ていますが,これは単に不統一なだけで,深い意味はありません.

参考文献は,ふつうに \verb+\cite{foo}+ と書いておくと,「こ
れまでの研究\cite{foo}では\UTF{FF5E}」のようになります.以前の 
\verb+iscie.sty+ では当時の会誌や論文誌に合わせた形式となる
ように設定していましたが,最近の会誌は\LaTeX\ の標準形式に準
拠しており,\verb+sci.sty+ でも同様にしています.

論文原稿の投稿後,学会側で余白に脚注などが追加される可能性を
考慮すると,\verb+\footnote+ は使用しないほうが無難かと思わ
れます.

節の区切りでは,\verb+\section+ や \verb+\subsection+ などを
使うことを前提として,文字サイズや行間隔などを再定義していま
す.必要に応じて,該当部分の定義を調整してください.

原稿中に明示的な文字サイズ変更指定を含めることは想定していま
せん.したがって,本文中で \verb+\Large+ や \verb+\small+ な
どを使用した場合には,文字サイズや改行間隔などが必ずしも意図
どおりにならない可能性があります.

\section{おわりに}

このスタイルファイルには,改善を要する点が多数含まれているこ
とと存じますが,会員諸賢によって,よりよいものが作成されるた
めの足がかりとなれば幸いです.

\renewcommand{\refname}{参考文献}
\begin{thebibliography}{2}
\bibitem{foo}
I. S. Cie: The ISCIE Style File;
ISCIE Journal, Vol.~0, No.~0, pp.~000--999 (2000)
\bibitem{bar}
文献:ISCIE用スタイルファイル;
システムと制御と情報,Vol.~00, No.~0, pp.~000--999 (2000)
\end{thebibliography}
\end{document}
% end of sci.tex
